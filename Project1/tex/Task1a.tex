\documentclass[oneside,onecolumn,9pt]{article}
\usepackage[english]{babel}

\usepackage{relsize,makeidx,color,setspace,amsmath,amsfonts,amssymb}
\usepackage[table]{xcolor}
\usepackage{bm,ltablex,microtype}

\usepackage[pdftex]{graphicx}

\usepackage[T1]{fontenc}
%\usepackage[latin1]{inputenc}
\usepackage{ucs}
\usepackage[utf8x]{inputenc}

\usepackage{lmodern}         % Latin Modern fonts derived from Computer Modern

% Hyperlinks in PDF:
\definecolor{linkcolor}{rgb}{0,0,0.4}
\usepackage{hyperref}
\hypersetup{
    breaklinks=true,
    colorlinks=true,
    linkcolor=linkcolor,
    urlcolor=linkcolor,
    citecolor=black,
    filecolor=black,
    %filecolor=blue,
    pdfmenubar=true,
    pdftoolbar=true,
    bookmarksdepth=3   % Uncomment (and tweak) for PDF bookmarks with more levels than the TOC
    }
%\hyperbaseurl{}   % hyperlinks are relative to this root

\setcounter{tocdepth}{2}  % levels in table of contents

% --- fancyhdr package for fancy headers ---
\usepackage{fancyhdr}
\fancyhf{} % sets both header and footer to nothing
\renewcommand{\headrulewidth}{0pt}
%\fancyfoot[LE,RO]{\thepage}
% Ensure copyright on titlepage (article style) and chapter pages (book style)
%\fancypagestyle{plain}{
%  \fancyhf{}
%  \fancyfoot[C]{{\footnotesize \copyright\ 1999-2019, "Computational Physics I FYS4411/FYS9411":"http://www.uio.no/studier/emner/matnat/fys/FYS4411/index-eng.html". Released under CC Attribution-NonCommercial 4.0 license}}
%%  \renewcommand{\footrulewidth}{0mm}
%  \renewcommand{\headrulewidth}{0mm}
%}
%% Ensure copyright on titlepages with \thispagestyle{empty}
%\fancypagestyle{empty}{
%  \fancyhf{}
%  \fancyfoot[C]{{\footnotesize \copyright\ 1999-2019, "Computational Physics I FYS4411/FYS9411":"http://www.uio.no/studier/emner/matnat/fys/FYS4411/index-eng.html". Released under CC Attribution-NonCommercial 4.0 license}}
%  \renewcommand{\footrulewidth}{0mm}
%  \renewcommand{\headrulewidth}{0mm}
%}

\pagestyle{fancy}


% prevent orhpans and widows
\clubpenalty = 10000
\widowpenalty = 10000

% --- end of standard preamble for documents ---


% insert custom LaTeX commands...

\raggedbottom
\makeindex
\usepackage[totoc]{idxlayout}   % for index in the toc
\usepackage[nottoc]{tocbibind}  % for references/bibliography in the toc

%-------------------- end preamble ----------------------

\begin{document}

% matching end for #ifdef PREAMBLE

\newcommand{\exercisesection}[1]{\subsection*{#1}}


% ------------------- main content ----------------------



% ----------------- title -------------------------

\thispagestyle{empty}

\begin{center}
{\LARGE\bf
\begin{spacing}{1.25}
Project 1, deadline  March 22 
\end{spacing}
}
\end{center}

% ----------------- author(s) -------------------------

\begin{center}
{\bf \href{{https://www.uio.no/studier/emner/matnat/fys/FYS4411/v19/index.html}}{Computational Physics II FYS4411}}
\end{center}

\begin{center}
% List of all institutions:
\centerline{{\small Department of Physics, University of Oslo, Norway}}
\end{center}
    
% ----------------- end author(s) -------------------------

% --- begin date ---
\begin{center}
\today
\end{center}
% --- end date ---

\vspace{1cm}

%%%%%%%%%%%%%%%%%%%%%%%%%%

%%%  --- begin 1a ---

\paragraph{Task: 1a}
The trial wave function, Eq. (\ref{eq:TrialWf}), for the ground state with $N$ atoms is given by

\begin{equation}
 \Psi_T(\mathbf{r})=\Psi_T(\mathbf{r}_1, \mathbf{r}_2, \dots \mathbf{r}_N,\alpha,\beta)=\prod_i g(\alpha,\beta,\mathbf{r}_i)\prod_{i<j}f(a,|\mathbf{r}_i-\mathbf{r}_j|),
 \label{eq:TrialWf}
 \end{equation}
 where $\alpha$ and $\beta$ are variational parameters. The
 single-particle wave function is proportional to the harmonic
 oscillator function for the ground state, i.e.,
 
 Want to find the analytic expressions for the local energy
\begin{equation}
    E_L(\mathbf{r})=\frac{1}{\Psi_T(\mathbf{r})}H\Psi_T(\mathbf{r})
    \label{eq:LocalE}
\end{equation}
 for the trial wave function of Eq. (\ref{eq:TrialWf}). 

\paragraph{Solution: Task 1 a}
\subparagraph{Non-interacting: 1D}

\begin{equation}
 \Psi_T(\mathbf{x})=\Psi_T(\mathbf{x}_1, \mathbf{x}_2, \dots \mathbf{x}_N,\alpha,\beta)=\prod_i g(\alpha,\beta,\mathbf{x}_i)= \exp{-\alpha(x_i^2)}
 \label{eq:TrialWfN1D}
 \end{equation}
 
 \begin{equation}
\frac{\partial \Psi_T(\mathbf{x})}{\partial x}= -2 \alpha x_i \exp{-\alpha(x_i^2)}
 \label{eq:dTrialWfN1D}
 \end{equation}
 
  \begin{equation}
\frac{\partial^2 \Psi_T(\mathbf{x})}{\partial^2 x}= 2\alpha (2\alpha x_i^2-1) \exp{-\alpha(x_i^2)}
 \label{eq:ddTrialWfN1D}
 \end{equation}
 
 \begin{equation}
    E_L(\mathbf{x})=\frac{1}{\Psi_T(\mathbf{x})}H\Psi_T(\mathbf{x}) = \frac{1}{\exp{-\alpha(x_i^2)}} 
   [ \alpha(1-2\alpha x_i^2) + ( \frac{1}{2}m\omega_{ho}^2x_i^2)] \exp{-\alpha(x_i^2)} 
\label{eq:LocalEN1D}
\end{equation}

 \begin{equation}
    E_L(\mathbf{x})= \prod_i  \alpha(1-2\alpha x_i^2) + ( \frac{1}{2}m\omega_{ho}^2x_i^2)
\label{eq:LocalEN1D2}
\end{equation}
 
\subparagraph{Non-interacting: 2D}

\begin{equation}
 \Psi_T(\mathbf{r})=\Psi_T(\mathbf{x}_1\mathbf{y}_1, \mathbf{x}_2\mathbf{y}_2 \dots \mathbf{x}_N\mathbf{y}_N,\alpha,\beta)=\prod_i g(\alpha,\beta,\mathbf{x}_i\mathbf{y}_i) = \exp{-\alpha(x_i^2+y_i^2)}
 \label{eq:TrialWfN2D}
 \end{equation}

 \begin{equation}
    E_L(\mathbf{r})
    = \frac{1}{\exp{-\alpha(r_i^2)}} 
    [ -\frac{1}{2} (4 \alpha (\alpha (r_i^2) -1) \exp -\alpha (r_i^2) + \frac{1}{2}m\omega_{ho}^2r_i^2 \exp{-\alpha(r_i^2)}] 
\label{eq:LocalEN2D}
\end{equation}

 \begin{equation}
    E_L(\mathbf{r})
    =  -2 \alpha (\alpha (r_i^2) -1) + \frac{1}{2}m\omega_{ho}^2r_i^2
    = \prod_i   \alpha (2-2\alpha (x_i^2+y_i^2) + \frac{1}{2}m\omega_{ho}^2(x_i^2+y_i^2)
\label{eq:LocalEN2D2}
\end{equation}


\subparagraph{Non-interacting: 3D}

\begin{equation}
 \Psi_T(\mathbf{r})=\Psi_T(\mathbf{r}_1, \mathbf{r}_2, \dots \mathbf{r}_N,\alpha,\beta)=\prod_i g(\alpha,\beta,\mathbf{r}_i) = \exp{-\alpha(x_i^2+y_i^2+\beta z_i^2)}
 \label{eq:TrialWfN3D}
 \end{equation}

 \begin{equation}
    E_L(\mathbf{r}) = \frac{1}{\exp{-\alpha(r_i^2)}} [ \frac{1}{2} (2 \alpha(2 \alpha r_i^2 -3) \exp -\alpha r_i^2) + \frac{1}{2}m\omega_{ho}^2r_i^2 \exp{-\alpha(r_i^2)}]
\label{eq:LocalEN3D}
\end{equation}

 \begin{equation}
    E_L(\mathbf{r}) = \prod_i  \alpha(3-2 \alpha r_i^2) + \frac{1}{2}m\omega_{ho}^2r_i^2
\label{eq:LocalEN3D2}
\end{equation}


%%%

\paragraph{Task: 1a drift force}
The drift force of Eq. (\ref{eq:DriftForce}) can be expressed by 
\begin{equation}
F = \frac{2\nabla \Psi_T}{\Psi_T}
\label{eq:DriftForce}
\end{equation}

Want to compute the analytical expression for the drift force to be used in importance sampling. 

\paragraph{Solution: 1a drift force}

\subparagraph{Non-interacting: 1D}

\begin{equation}
F = \frac{2\nabla \Psi_T}{\Psi_T}= \frac{2 [(-2 \alpha x_i^2) \exp-\alpha x_i^2]}{\exp-\alpha x_i^2} 
= -4 \alpha x_i^2
\label{eq:DriftForceN1D}
\end{equation}

\subparagraph{Non-interacting: 3D}

\begin{equation}
F = \frac{2\nabla \Psi_T}{\Psi_T}= \frac{2 [(-2 \alpha r_i^2) \exp-\alpha r_i^2]}{\exp-\alpha r_i^2} 
= -4 \alpha r_i^2
\label{eq:DriftForceN3D}
\end{equation}

%%% 


%%%  --- end 1a ---


%%%%%%%%%%%%%%%%%%%%%%%%


% ------------------- end of main content ---------------

\end{document}
